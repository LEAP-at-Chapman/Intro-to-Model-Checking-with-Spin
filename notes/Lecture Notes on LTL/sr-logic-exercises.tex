\section{Questions and Exercises}
Questions marked with * relate to material relevant for the
exam. Questions of particular importance for the exam are marked with
**.
\begin{question}* \ 
  A traffic light can be modelled in propositional logic by three
  atomic propositions red, yellow, green. Without further
  specification, this means that a traffic light can be in $2^3=8$
  different states. Now suppose you want to model a crossing with 4
  traffic lights (for cars) and 4 traffic lights for cyclists (the
  latter have only red and green). How many possible states are there
  for the crossing now? What if you add 8 more traffic lights (two
  colours each) for pedestrians? Explain the notion of `state
  explosion'.
\end{question}

\begin{question}* \
For each of the following formulae determine whether they are valid
and whether they are satisfiable. 
\begin{center}
\begin{tabular}{c}
$p\imp p$\\
$p\et\neg p$\\
$p\et\neg q$\\
$p\ou\neg p$\\
$p$\\
$\neg p$
\end{tabular}
\end{center}

\end{question}

\begin{question}* \
  List all models of the following traffic light specification
  ($r,y,g$ for red, yellow, green).
\[
(r\ou y)\imp \neg g
\]
Do this using a truth table and do this again using a semantic
tableau. Compare the two procedures.  Add a formula that makes sure
that not all colours can be dark simultaneously.
\end{question}
 
\begin{question}
\begin{enumerate}
\item
  Explain in what sense propositional logic can also be defined by the
  smaller grammar
%
$$\phi ::= p \mid  \neg\phi \mid \phi\et\phi$$
%
 [Hint: Use the equivalences of Example~\ref{exle:prop:valid}.]
\item Similarly, show that from each of the following sets of
  connectives $\{\bot,\imp\}$, $\{\neg,\ou\}$ all other connectives
  can be defined. 
\end{enumerate}
\end{question}

\begin{question}** \
Use semantic tableaux to check whether the following formulae are valid
and, in case they are not, give a counterexample.
\begin{center}
\begin{tabular}{c}
$((p\et q)\imp r) \imp (p\imp (q\imp r))$\\ %v
$(p\imp (q\imp r))\imp ((p\et q)\imp r)$\\  %v
$((p\et q)\imp r) \imp ((p\imp q)\imp r)$\\ %nv
$((p\imp q)\imp r)\imp ((p\et q)\imp r)$\\  %v
%$(p\imp (q\ou\neg r))\et p\et r \imp q$\\
$(p\et q \et (q\imp( r\imp p)))\imp r$\\ %nv
%$(p \et \neg(p\et q) \et (\neg r\imp q) \et (r\imp q)) \imp s$\\
$(r\et q) \imp (((r\imp s)\et q) \ou \neg(q\imp s))$ \\ %(0607)
$(p\imp (q\imp r))\imp((p\imp q)\imp (p\imp r))$ \\%(0708)
%some new ones:
% $(p\imp(p\et r))\ou(r\imp(p\et\neg p))$\\%valid
% $(p\ou q \imp r )\ou((r\imp p)\et(r\imp q))$\\%valid
% $(p\imp(q\ou r))\ou\neg(\neg q\imp r)$\\%valid
% $(p\et q)\ou(r\et q)\ou\neg q\ou (\neg p \et q \et r)$\\%valid
% $\neg q\imp(\neg(p\et q)\imp(r\et q))$\\%not valid  
\end{tabular}
\end{center}
\end{question}


\begin{question} \ \label{qn:traffic-light-tableau} 
\begin{enumerate}
\item
Show that the following FOL-formulae are valid.
\begin{center}
\begin{tabular}{c}
$\forall x.\phi \imp \exists x.\phi$\\
$\exists y.\forall x.\phi\imp \forall x.\exists y. \phi$\\
$\forall x.(\phi\imp \psi) \imp(\forall x. \phi \imp \forall x. \psi)$\\
\end{tabular}
\end{center}
\item
Give counter-examples for 
\begin{center}
\begin{tabular}{c}
$\forall x.\exists y. \phi \imp \exists y.\forall x.\phi$\\
$(\forall x. \phi \imp \forall x. \psi) \imp \forall x. (\phi\imp \psi) $\\
\end{tabular}
\end{center}
\end{enumerate}
\end{question}

\begin{question}
  Why is PL decidable. Why does this argument not apply to FOL
  (First-order predicate logic)? Is FOL decidable?
\end{question}


\begin{question}* \
Define 
\begin{itemize}
\item $M,n\models \phi\atnext\psi$ \ if \ \
  \parbox{20em}{$M,m\notmodels\psi$ for all $m> n$ or \\
    $M,k\models\phi$ for the smallest $k> n$ with $M,k\models\psi$.}
\item $M,n\models \phi\while\psi$ \ if \ \
  \parbox{20em}{$M,m\notmodels\psi$ for some $m > n$ and\\
    $M,k\models\phi\et\psi$ for all $k$ with $n< k< m$.}
\item $M,n\models \phi\before\psi$ \ if \ \
  \parbox{20em}{for all $m> n$ with $M,m\models\psi$\\
                there is  $k$ with $n< k< m$ such that $M,k\models\phi$.}
\end{itemize}
Express these operators using LTL.
\end{question}

\begin{question}** \
  For each of the following LTL-formulae, either show that the formula
  is valid or give a counterexample.
\begin{enumerate}
\item $\next\Box p \imp \Box\next p$,
\item $\Box(p\ou q) \imp \Box p \ou \Box q$,
\item $\Box p \ou \Box q \imp \Box(p\ou q)$,
\item $\Box\Diamond\phi\imp\Diamond\Box\phi$,
\item $\Box(\phi\imp\next\Diamond\psi) \et \Diamond \phi
  \imp\Diamond\psi$.
\item $\Box q \et (p\until q) \imp \Box p$,
\item $\next\phi \biimp \bot\until\phi$.
\item
\item $\Box(p\imp\Diamond\neg p)\biimp \Box\Diamond\neg p$.
\end{enumerate}
\end{question}

\begin{question}* \
  Further temporal laws: Show that the following are valid.
\begin{enumerate}
\item $\Box(p\imp \next q) \wedge p \imp \Box q$. \ [This law is a
  powerful induction principle if you want to prove that a formula of
  the kind $\Box\phi$ is true.]
\item $(\Box\Diamond p \imp \Box\Diamond q) \ \imp \ \Box(\Box\Diamond
  p \imp \Box\Diamond q)$. \ [This is related to fairness.]
\end{enumerate}
\end{question}

\begin{question} * \
\begin{enumerate}
\item Give a counterexample to $p\until q \ \rightarrow p\before q$.
\item Change the definition of the semantics of $\until$ and $\before$
  in such a way that the formula above becomes valid.
\end{enumerate}
\end{question}

\end{document}


%further questions
\begin{question} 
\begin{enumerate}
\item Give the semantics of the $\until$-operator of LTL as in the
  lecture.
\item Give the semantics of the $U$ operator of SPIN.
\item Comparing $\until$ and $U$, what are the respective advantages
  of the two. 
\end{enumerate}

Define $p\before q$ as $\neg(\neg p \until q)$ and $pBq$ as $\neg(\neg
p U q)$.
\begin{enumerate}
\item Give a counterexample to $p\until q \ \rightarrow
  p\before q$.
\item Argue that $p U q \ \rightarrow \ pBq$ is valid.
\end{enumerate}
\end{question}
