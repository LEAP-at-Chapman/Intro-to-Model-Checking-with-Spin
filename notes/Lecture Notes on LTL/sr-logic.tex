\documentclass[11pt]{article} 
\usepackage{fullpage}
\usepackage{verbatim}
%\usepackage{html}
\usepackage{url}
\usepackage{hyperref}
\usepackage{bbm}
\usepackage{amsmath}
\usepackage{amsthm}
\usepackage{txfonts}
\usepackage{amsfonts}
\usepackage[all]{xy}

%\input{logic}
\newcommand{\sth}{\ensuremath{{\;|\;}}} 
\renewcommand{\phi}{\varphi} 
%%% first order shorthands
\newcommand{\imp}{\ensuremath{\rightarrow}}
\newcommand{\biimp}{\ensuremath{\leftrightarrow}}
\newcommand{\et}{\ensuremath{\wedge}} %\and is already a latex command
\newcommand{\Et}{\ensuremath{\bigwedge}} %\and is already a latex command
\newcommand{\ou}{\ensuremath{\vee}} %\or is already a tex command
\newcommand{\Ou}{\ensuremath{\bigvee}} %\or is already a tex command
\newcommand{\fa}{\ensuremath{\forall}}
\newcommand{\ex}{\ensuremath{\exists}}
\newcommand{\eq}{\ensuremath{\leftrightarrow}}
\newcommand{\equ}{\ensuremath{\eq}}
%%%***%%% metalogical shorthands
\newcommand{\deriv}{\ensuremath{\mathrel{\vdash}}}
%\newcommand{\notderiv}{\ensuremath{\mathrel{\mathpalette\notthrmm\thrm}}}
  %cf. the def. of \notthrm
\newcommand{\notmodels}{\ensuremath{\mathrel{\mathpalette\notmodelss\models}}}
  \newcommand{\notmodelss}[2]{\ooalign{$\hfil#1\mkern1mu/\hfil$\crcr$#1#2$}}
  %cf. the def. of \notin in plain.tex
\newcommand{\modelsb}{\ensuremath{\mathrel{\mathpalette\modelsbb\models}}}
\newcommand{\modelsbb}[2]{\ooalign{$\hfil#1\mkern3mu\models\hfil$\crcr$#1#2$}}
\newcommand{\modelsbp}{\modelsb}
%%% small connectives
\newcommand{\oif}{\ensuremath{\;\Rightarrow\;}} % only if
\newcommand{\si}{\ensuremath{\;\Leftarrow\;}} % if
\newcommand{\ssi}{\ensuremath{\;\Leftrightarrow\;}} % iff (si et seulement si)
\newcommand{\y}{\ensuremath{\;\&\;}} % and
%%% large connectives
\newcommand{\Oif}{\ensuremath{\:\;\Longrightarrow\;\:}} % big only if
%\newcommand{\Si}{\ensuremath{\:\;\Longleftarrow\;\:}} % big if
\newcommand{\Ssi}{\ensuremath{\;\;\Longleftrightarrow\;\;}} % big iff (si et seulement si)
%end logic.tex


% \setlength{\parindent}{0pt} 
% \setlength{\parskip}{1ex plus 0.5ex minus 0.2ex} 

\newcommand{\sem}[1]{[\![#1]\!]} 
\newcommand{\bbtwo}{\mathbbm{2}}
\newcommand{\bbP}{\mathbbm{P}}
\newcommand{\until}{\mathrel{\mathsf{until}}}
\newcommand{\atnext}{\mathrel{\mathsf{atnext}}}
\newcommand{\while}{\mathrel{\mathsf{while}}}
\newcommand{\before}{\mathrel{\mathsf{before}}}
\newcommand{\next}{\medcirc}
\newcommand{\bbN}{\mathbbm{N}}

%\newcommand{\comment}[1]{}

\newcommand{\rules}[2]{\mbox{$\frac% 
                      {\mbox{\normalsize \rule[-5pt]{0pt}{14pt} $#1$}} 
                      {\mbox{\normalsize \rule[0pt]{0pt}{10pt}$#2$}}$}} 

\newcounter{quest}
\newenvironment{question}{\refstepcounter{quest}\begin{trivlist}
\item[\hspace{\labelsep}\bf Question \thequest.]}{\end{trivlist}}

\newcounter{answ}
\newenvironment{answer}{\refstepcounter{answ}\begin{trivlist}
\item[\hspace{\labelsep}\bf Answer \theansw.]}{\end{trivlist}}

\newcounter{prop}
\newcommand{\newprop}[2]{\newenvironment{#1}{\refstepcounter{prop}
\begin{trivlist}\item[\hspace{\labelsep}\bf #2 \theprop .]}{\end{trivlist}}}
\newprop{definition}{Definition}
\newprop{defn}{Definition}
\newprop{prop}{Proposition}
\newprop{thm}{Theorem}
\newprop{lemma}{Lemma}
\newprop{example}{Example}
\newprop{exercise}{Exercise}
\newprop{remark}{Remark}
\newprop{notn}{Notation}
\newenvironment{pf}{\begin{trivlist}\item[\hspace{\labelsep}\bf Proof.]}
{\hfill$\square$\end{trivlist}}


\begin{document}
% \begin{center}
% \large\textbf{CO2014 Worksheet 1}
% \end{center}

\title{Linear Temporal Logic with Spin}
\author{Alexander Kurz}
\date{\today}
\maketitle
\tableofcontents
%\newpage

\newcommand{\Fma}{\mathsf{Fma}}

%\input{sr-logic1}%Propositional logic
\section{Propositional Logic}

\subsection{Truth Tables}

Notation: We write $\bbtwo$ for the two-element set $\{0,1\}$.  We
call the elements of $\bbtwo$ truth values and refer to 0 as `false'
and to 1 as `true'.

\bigskip\noindent The formulas of propositional logic (PL) are given
wrt\footnote{with respect to} a set $\bbP$ of \emph{atomic
  propositions}, sometimes also called \emph{propositional variables},
according to the context-free
grammar\footnote{\url{http://en.wikipedia.org/wiki/Context-free_grammar}, 
  \url{http://en.wikipedia.org/wiki/Backus-Naur_Form}}
%
  \[\Fma ::= p \mid \bot \mid \top \mid \neg\Fma \mid \Fma\ou\Fma \mid
  \Fma\et\Fma \mid \Fma\imp\Fma \mid \Fma\biimp\Fma \ \ \footnote{The
    connectives should be pronounced as false, true, not, or, and, implies,
    if and only if. The connectives are named: falsum/bottom,
    verum/top, negation, disjunction, conjunction, implication,
    biimplication (or logical equivalence).}\]
%
  where $p$ takes values in $\bbP$. We also say that this grammar
  discribes the \emph{syntax} of propositional logic. To save brackets
  we use the convention that $\neg$ binds stronger than $\et$, $\ou$
  which bind stronger than $\imp$ which binds stronger than
  $\biimp$.\footnote{For example, $\neg(\phi\ou\psi) \biimp
    \neg\phi\et\neg\psi$ abbreviates $(\neg(\phi\ou\psi)) \biimp
    ((\neg\phi)\et(\neg\psi))$.}

\bigskip\noindent\emph{Remark on Notation: } $\Fma$ is a so-called
non-terminal symbol of the context-free grammar describing all
propositional formulas. We use $\phi$ to denote a propositional
formula. We use $p,q,r$ to denote atomic propositions.

\newcommand{\red}{\mathsf{red}}
\newcommand{\yellow}{\mathsf{yellow}}
\newcommand{\green}{\mathsf{green}}

\begin{example}
  Intuitively, propositional logic is the logic of finite states, or,
  in other words, the
  logic to describe static properties of a system with finite memory. Consider,
  eg \[\bbP=\{\red,\yellow,\green\}.\] Then we may use propositional
  formulas to specify static properties of traffic lights such as
\begin{itemize}
\item $\red\imp\neg\green$
\item $\green\imp\neg\red$
\item $\neg\red\ou\neg\green$
\end{itemize}
(One thing propositional logic cannot do, is to specify dynamic, or
\emph{temporal}, properties, such as: whenenver the traffic light is
red then \emph{eventually} it will become green. We will come back to
this when we discuss temporal logic.) 
\end{example}

\medskip\noindent A good exercise at
this point is too write out a full specification of the allowed
states of a traffic light. And then to ask: How we can
validate this specification?

\medskip\noindent This exercise will lead to some interesting
questions about specifications. For example, when are two different
specifications equivalent?

\bigskip These questions lead us to the notion of model of a
specification.\footnote{Here we use the word model as a technical term
  in the sense of logic as made precise below. It is important to
  understand this technical meaning and not to confuse it with the
  meaning this word has in physics and engineering, or also in some
  areas of software engineering (model driven architectures,
  metamodels, \ldots)} Intuitively, a model of a specification
describes a situation in which the specification is true. For
example, the models of  $\red\imp\neg\green$ are all combinations a
traffic light could show as long as red and green are not both on. For
example 
\begin{gather*}
(\red\mapsto 1, \yellow\mapsto 1, \green\mapsto 0) \\
(\red\mapsto 0, \yellow\mapsto 1, \green\mapsto 1) \\
(\red\mapsto 0, \yellow\mapsto 0, \green\mapsto 0)
\end{gather*}
are all models of $\red\imp\neg\green$, but
\begin{gather*}
(\red\mapsto 1, \yellow\mapsto 1, \green\mapsto 1) \\
(\red\mapsto 0, \yellow\mapsto 0, \green\mapsto 1) 
\end{gather*}
are not. After this informal explanation, let us give a formal
definition. We first define the notion of model in general and come
back to the notion of a model of a specification later.

\begin{definition}
Given the set of atomic propositions  $\bbP$, a \emph{model} is map,
  sometimes called a valuation, $v:\bbP\to\mathbbm{2}$.
\end{definition}

The reason, the notion of a model is defined as it is above, is that a
model is exactly what is needed in order to assign a truth value to a
formula. For example, to say that
\[p\imp q\]
is true we need to know what $p$ and $q$ is \ldots actually, it is enough
to know what the truth values of $p$ and $q$ are \ldots and that is
exactly what a model tells us. I assume everybody knows how to evaluate
propositional formulas, so let us go directly to a formal description
of the algorithm for evaluating propositional formulas given truth
values of the atomic propositions.

\medskip\noindent $v$ is extended from atomic propositions to formulas
using the \emph{truth tables} corresponding to the connectives $\bot,
\top, \neg, \ou, \et, \imp, \biimp$. If we write 0, 1,
\boldmath$\neg$\unboldmath, \boldmath \et\unboldmath, \boldmath
\ou\unboldmath, \boldmath \imp\unboldmath,\boldmath \biimp\unboldmath,
for the respective truth tables\footnote{Eg the truth tables for `and'
  \quad
\begin{tabular}{|l|l|c|} % chktex 44
\hline
$p$ & $q$ & $p$\boldmath\et\unboldmath$q$\\
\hline
\hline
0&0&0\\
\hline
0&1&0\\
\hline
1&0&0\\
\hline
1&1&1\\
\hline
\end{tabular}
\quad and implication \quad
\begin{tabular}{|l|l|c|}
\hline
$p$ & $q$ & $p$\boldmath\imp\unboldmath$q$\\
\hline
\hline
0&0&1\\
\hline
0&1&1\\
\hline
1&0&0\\
\hline
1&1&1\\
\hline
\end{tabular}. 
If you have doubts about the first two lines of the truth-table for the
implication consider ($x$ is even and $y$ is odd)$\imp$($x$ is even)
and evaluate it for $(x,y)$ taking the values $(3,2)$ and $(2,2)$.
}\label{fn:truth-tables} then
\begin{align*}
\sem{p}_v & = v(p)\\
\sem{\bot}_v & = 0\\
\sem{\top}_v & = 1\\
\sem{\phi\et\psi}_v & = \sem{\phi}_v\mbox{\boldmath
\et\unboldmath}\sem{\psi}_v\\
\sem{\phi\ou\psi}_v & = \sem{\phi}_v\mbox{\boldmath
\ou\unboldmath}\sem{\psi}_v\\
\sem{\phi\imp\psi}_v & = \sem{\phi}_v\mbox{\boldmath
\imp\unboldmath}\sem{\psi}_v\\
\sem{\phi\biimp\psi}_v & = \sem{\phi}_v\mbox{\boldmath
\biimp\unboldmath}\sem{\psi}_v\\
\end{align*}

\medskip\noindent Instead of formalising ``meaning'' as a function $\sem{-}_-$
from formulas and models to truth values, we can also formalise the
relation between models and formulas saying when a formula holds (is
true) in a model. This relation is usually denoted by $\models$ or
$\Vdash$ and can be seen as a useful alternative, but equivalent,
notation. The two notations are related via  \[\sem{\phi}_v=1 \quad \textrm{if and only if} \quad
v\models\phi.\] A direct definition of $\models$:
\begin{align*}
  v\models p & \quad\textrm{ if }  v(p)\\
  v\models\top \\
  v\models{\neg\phi} & \quad\textrm{ if it is not the case that } v\models\phi \\
  v\models{\phi\et\psi} & \quad\textrm{ if } v\models\phi \textrm{ and } v\models\psi
\end{align*}
(the other logical connectives can be defined similarly.)

\begin{definition}
\begin{itemize}
\item Each of the following are equivalent ways of saying $\sem{\phi}_v=1$
\begin{itemize}
\item $v\models\phi$
\item $\phi$ \emph{holds} in $v$
\item $v$ satisfies $\phi$
\item $\phi$ is satisfied in $v$
\item $v$ is a model of $\phi$
\end{itemize}

\item $\phi$ is \emph{satisfiable} iff there is $v:\bbP\to \bbtwo$ such that
  $v\models\phi$.
\item $\phi$ is \emph{valid} iff for all $v:\bbP\to \bbtwo$ we have
  $v\models\phi$.
\item $\phi$ and $\psi$ are equivalent, written $\phi\equiv\psi$, iff
  for all $v:\bbP\to \bbtwo$ we have $\sem{\phi}_v=\sem{\psi}_v$.
\end{itemize}
\end{definition}

\begin{example}\label{exle:prop:valid} 
\begin{itemize}
\item The following formulas are
  valid \footnote{Valid formulas are also called tautologies. Formulas
  that are not satisfiable are also called contradictions.}
\begin{align*}
p\imp p\\
p\ou\neg p\\
p\imp(q\imp p)
\end{align*}
\item The following formulas are not satisfiable
\begin{align*}
p\et\neg p\\
(p\imp q)\et(p\imp\neg q)
\end{align*}
The following formulas satisfiable
\begin{align*}
p\\
\neg p
\end{align*}

\item The following are equivalences
\begin{align*}
(\phi\biimp\psi) & \equiv  (\phi\imp \psi)\et(\psi\imp \phi)\\
\neg(\phi\ou\psi)& \equiv \neg\phi\et\neg\psi\\
\neg(\phi\et\psi)& \equiv \neg\phi\ou\neg\psi\\
\phi\imp\psi &\equiv \neg\phi\ou\psi\\
\neg(\phi\imp\psi)&\equiv \phi\et\neg\psi
\end{align*}
This means that whatever formulas you plug in for $\phi$ and $\psi$,
and however you evaluate the atomic propositions in these formulas,
the left-hand side of ``$\equiv$'' will have the same truth value as
the right-hand side. Therefore, whenever you encounter a left-hand
side, you can replace it by the corresponding right-hand side (and
vice versa). This will be important in the construction of semantic
tableaux below. Also note that $\phi\equiv\psi$ iff 
$\phi\biimp\psi$ is valid.
\end{itemize}
\end{example}



\bigskip\noindent\emph{Remark on Notation: } One can think of formulas $\phi$
as programs and the function $\sem{-}$ as a compiler.  Then
$\sem{\phi}$ is the compiled program which computes on input $v$ the
output $\sem{\phi}_v$. This notation is used quite widely in computer
science and you may meet it in other places as well.

\bigskip\noindent\emph{Remark on Semantics: } The notation $\sem{-}$
is often called the semantic brackets, because $\sem{-}_v$ takes a
piece of syntax, namely a formula $\phi$, and maps it to the meaning
of the formula, which is the truth value $\sem{\phi}_v$ in a given
model $v$. To summarize, the \emph{semantics} (= meaning) of a formula
given a model $v$ is its truth value, and it is computed by
$\sem{-}_v$ using the truth tables. The \emph{semantics} (= meaning)
of a formula is the set of models satisfying the formula. Or, to say
the same in software-engineering jargon, the semantics of a specification is
the set of implementations satisfying the specification. This leads to:

\bigskip\noindent\emph{Second Remark on Semantics: } The situation in
logic is simpler, but completely analogous, to what we have for
programming languages. Indeed a program, like a formula, is a piece of
syntax, built according to the rules of a context free grammar. If the
program is written in an imperative programming language such as C or
Java, then we can take the semantics of the program to be given, for
example, by the compiler which determines how the program actually
modifies the memory.

\medskip\noindent\textbf{Question: } What are the models for $p\imp
q$? How can you find the models of a formula $\phi$ by looking at the
truth table of $\phi$?




\subsection{An algorithm for satisfiability: semantic tableaux}

We know how to prove the validity of a formula using truth tables.
But truth tables grow exponentially with the number of atomic
propositions involved (if $n$ is the number of propositions then $2^n$
is the number of rows of the truth table). In this section we learn
another method which is often in practice (but not in the worst case)
more efficient and is the basis of many model-checking algorithms:
semantic tableaux. In fact, we can think of the semantic tableau for a
formula as a systematic search for all models that satisfy the
formula. The relationship between validity and satisfiability is as
follows.

\medskip\noindent{\Large \textbf{Important Fact. } 
\quad $\phi$ is valid \ iff \footnote{if and only if} \  $\neg\phi$ is not satisfiable.}

\medskip\noindent Equivalent is the following, also sometimes useful:
\ $\neg\phi$ is valid iff $\phi$ is not satisfiable.

\medskip A \emph{complete semantic tableau} (for propositional logic)
is a tree \footnote{A typical tree looks like this
\[
\xymatrix{
&a\ar[dl]\ar[d]\ar[dr]&\\
b \ar[d] & c & d\ar[dl]\ar[d]\\
e\ar[d] & f &g\\
h
}
\] 
where nodes are labelled here with letters $\{a,\ldots h\}$. We say
that $a$ is the root; and $b,c,d$ are the children or successors of
$a$; and $h,c,f,g$ are the leaves; and
$(a,b,e,h),(a,c),(a,d,f),(a,d,g)$ are the branches. Examples of trees
abound in computer science. For example, each formula of propositional
logic is a tree, in which case the labels would be logical connectives
$\{\neg,\et,\ou,\imp,\biimp\}$ for non-leaf nodes and
$\bbP\cup\{\top,\bot\}$ for leaves. A semantic tableau will also be a
tree, in which case each node is labelled with a set of formulas.}
where a node is a set of formulas and the leaves only contain atomic
propositions or negations of atomic propositions.  Moreover, after
applying the equivalences of Example~\ref{exle:prop:valid} from left
to right, each non-leaf node $\Phi$ has to satisfy one of the
conditions of Table~\ref{table:tableau-rules}.

\begin{table}
\fbox{\parbox{\textwidth}{
\begin{itemize}
\item $\Phi$ contains a formula $\phi \ou\psi$ and has exactly
  two children given by $(\Phi\cup\{\phi\})\setminus\{\phi \ou\psi\}$
  and $(\Phi\cup\{\psi\})\setminus\{\phi \ou\psi\}$.
    
\item $\Phi$ contains a formula $\phi \et\psi$ and has exactly
  one child given by $(\Phi\cup\{\phi,\psi\})\setminus\{\phi
  \et\psi\}$.
      
\item $\Phi$ contains a formula $\neg\neg\phi$ and has exactly one
  child given by $(\Phi\cup\{\phi\})\setminus\{\neg\neg\phi\}$.  
\end{itemize}}}
\caption{Tableau rules}\label{table:tableau-rules}
\end{table}

Before we explain how to conclude satisfiability and validity from a
complete semantic tableau, let us first define some useful notions.

\begin{definition}
A node in a semantic tableau is closed if it contains a contradiction,
that is, if it contains a formula $\phi$ and its negation
$\neg\phi$. A leaf is called open if it is not closed. A branch in a
semantic tableau is closed if it contains a closed node. A semantic
tableau is closed if all branches are closed. Otherwise it is called open.
\end{definition} 

\medskip\noindent \textbf{A formula $\phi$ is valid if there is some closed semantic tableau
which has a root labelled by $\{\neg\phi\}$.}
(\textit{Remark: } If there is one closed semantic tableau for $\psi$
then all complete semantic tableaux of $\psi$ are closed. This
property is important: it shows that it doesn't matter which tableau
you construct.\footnote{But some tableaux may be bigger than others.
  For example, it makes sense, in case we have a choice of several
  rules, to apply the non-branching rules first.}) 

\medskip\noindent \textbf{A formula $\phi$ is satisfiable if there is some complete semantic tableau
which has (at least) one open leaf and  a root labelled by
$\{\phi\}$.}


\medskip The three rules of Table~\ref{table:tableau-rules} can be written more suggestively as in
Table~\ref{table:tableau-rules-short}.
\begin{table}
\fbox{\parbox{\textwidth}{
\[
\rules{\Gamma,\phi\ou\psi }{\Gamma,\phi \quad\quad \Gamma,\psi}
\quad\quad\quad
\rules{\Gamma,\phi\et\psi}{\Gamma,\phi,\psi}
\quad\quad\quad
\rules{\Gamma,\neg\neg\phi}{\Gamma,\phi}
\]
}}
\caption{Tableau rules, short form}\label{table:tableau-rules-short}
\end{table}
For example, in the left-hand rule, one has a parent node consisting
of a set $\Gamma$ of formulas and one additional formula
$\phi\ou\psi$; this node then has two children, the left of which
consists of $\Gamma$ and of $\phi$. Note that the formula
$\phi\ou\psi$ does not appear in any of the children. Also note how
each of the rules in Table~\ref{table:tableau-rules-short} corresponds
to one of the bullet points of Table~\ref{table:tableau-rules}.



\paragraph{Open leaves are counter-models } The method of semantic
tableaux is a systematic search for \emph{models} (that is why it is
called `semantic'). A closed leaf indicates that on this branch we
cannot find a model (since a closed leaf contains a contradiction and
no model can make a contradiction true). On the contrary, an open leaf
represents a set of models for the formula at the root of the tableau.

\medskip\noindent \emph{For example}, if we start a tableau with
$\{p\imp q\}$, we obtain two leaves, namely $\{\neg p\}$ and $\{q\}$,
which correspond to the three different models $v$ for which we have
$v\models p\imp q$. Compare this with the truth table of
Section~\ref{fn:truth-tables}.
% See Question~\ref{qn:traffic-light-tableau} for an exercise on this
% point of view.

\medskip\noindent\emph{Checking of the Countermodel} is what is
required in the typical exercises: If your tableau leads to an open
leaf, then extract a model and evaluate the original formula. If the
outcome is as expected, then you are done, otherwise there must be a
mistake in the tableau that you then can correct.


\paragraph{Summary } Question: Is $\phi$ valid? Start a tableau with
root $\{\neg\phi\}$. If you find an open leaf, you can stop, check
your counter-example and conclude that $\phi$ is not valid. If you
find no open leaf, ie all leaves are closed, then the tableau is
closed and $\phi$ is valid.

%\newpage

\section{Temporal Logic}

Predicate logic is a powerful logic, but not decidable in
general.\footnote{Knowledge of predicate logic is not required for
  these notes. But because of its importance in general and for a
  deeper understanding of model checking, we recall the basic
  definitions in an appendix.}  In this section, we will look at
special kinds of models and logics that have a certain form of
quantification but are still decidable. The general idea is, on the
one hand, to look at special models (transition systems, execution
sequences) and on the other hand, to only allow restricted use of
quantification (eg always, sometimes).


%\input{sr-logic-ltl}%LTL
\subsection{(Linear) Temporal Logic}\label{sec:LTL:def}

Linear temporal logic (LTL) is interpreted over infinite sequences,
also called runs or execution sequences. `Linear' emphasises that the
logic speaks about sequences and not, for example, about trees. The
semantics of LTL wrt sequences can be extended to a semantics wrt
transition systems by considering all execution sequences generated by
the transition system.

%%%%%\subsubsection{Definition of LTL}\label{sec:LTL:def}

\bigskip\noindent\textbf{Syntax of LTL\@. } $\Fma::=p \mid \neg\Fma \mid
\Fma\et\Fma \mid \next \Fma \mid \Box\Fma \mid \Diamond\Fma \mid
\Fma\until\Fma$
 
\medskip\noindent\textit{Precedence rules. } To save brackets, we say
that the unary operators bind stronger than $\until$ which binds
stronger than $\et,\ou$ which bind stronger than $\imp$ which binds
stronger than $\biimp$. For example, \[\neg p\et \Diamond q\until \Box
r \et s \imp t \biimp u\]
is bracketed as \[((\neg p\et ((\Diamond
q)\until \Box r) \et s) \imp t) \biimp u\]

\bigskip\noindent\textbf{Semantics of LTL\@. } Let $M=(v_n)_{n\in\bbN}$
be a sequence\footnote{$\bbN=\{0,1,2\ldots\}$ denotes the set of
  natural numbers; $(v_n)_{n\in\bbN}$, or $(v_n)$, is short
  for $(v_0,v_1,v_2\ldots)$.} of valuations $v_n:\bbP\to\bbtwo$.
\begin{itemize}
\item $M,n\models p$ \ if $v_n(p)=1$,
\item $M,n\models \neg \phi$ \ if not $M,n\models\phi$,
\item $M,n\models \phi\et\psi$ \ if $M,n\models\phi$ and
  $M,n\models\psi$,
\item $M,n\models\next\phi$ \ if $M,n+1\models\phi$,
\item $M,n\models\Box\phi$ \ if $M,m\models\phi$ for all $m\ge n$,
\item $M,n\models\Diamond\phi$ \ if $M,m\models\phi$ for some $m\ge n$,
\item $M,n\models \phi\until\psi$ \ if \ \
  \parbox{15em}{$M,m\models\psi$ for some $m> n$ and\\
    $M,k\models\phi$ for all $k$ with $n< k < m$.}
\end{itemize}

\noindent $M\models\phi$ \ if $M,n\models\phi$ for all $n\in\bbN$.\\
$M\models\Phi$ and $\Phi\models\phi$ and $\models\phi$ (`$\phi$ is
valid') as before.

\bigskip\noindent\textbf{Terminology. } Read \\
\renewcommand{\arraystretch}{1.4}
\hspace*{2em} \begin{tabular}{rcl}
$\next\phi$ && next $\phi$, \\
$\Box \phi$ & & always $\phi$,\\
$\Diamond\phi$ && sometimes $\phi$, or eventually $\phi$,\\ 
$\phi\until\psi$ && $\phi$ until $\psi$.
\end{tabular}

\bigskip\noindent\textbf{Remark. }
% \begin{enumerate}
% \item
[Only for readers of Appendix~\ref{sec:ml}] A model for LTL is a model
for ML where we take $\bbN$ as the carrier set and two relations. The
first relation is $\{(n,n+1)\mid n\in\bbN\}$ and interprets the
$\next$.\footnote{The box and the diamond for this relation are the
  same as each state $n$ has a unique successor; we therefore only
  need one operator and write it as $\next$.} The second relation is
$\le$ and interprets $\Box$ and $\Diamond$.
%\end{enumerate}

\bigskip\noindent\textbf{Some important formulae. }
%
\begin{center}
\begin{tabular}{|l|l|}
  \hline
  $\Box\Diamond p$ & infinitely often $p$\\
  \hline
  $\Diamond\Box p$ & eventually $p$ will always be true\\
  \hline
  $\neg (\neg p \until q)$ & p before q (see below) \\ 
  \hline
  $\Box(p\imp\Diamond q)$ &  progress: each `request' p gets `acknowledgement' q\\
  \hline
  $\neg\Diamond\Box p$ &  weak fairness: not forever `blocked at p'\\
  \hline
  $\Box\Diamond\neg p$ &  the same\\
  \hline
  $\Diamond\Box p\imp\Box\Diamond q$ &  weak fairness: `eventually always
  p (eg enabled) only if infinitely often q (eg executed)'\\
  \hline
  $\Box\Diamond p\imp\Box\Diamond q$ &  strong fairness: `infinitely
  often p (eg enabled) only if infinitely often q (eg executed)'\\
  \hline
\end{tabular}
\end{center}

\medskip\noindent Show that $\neg (\neg p \until q)$ means: whenever
in the future $q$ holds, then $p$ must have happened before $q$.

\medskip\noindent Fairness: Note that $\neg\Diamond\Box p$ is
equivalent to $\Diamond\Box p\imp\bot$, so it is a special case of the
more elaborate $\Diamond\Box p\imp\Box\Diamond q$. A typical example
of the latter could be: Always, if a process keeps waiting to enter and
the door remains open, then the process will (be scheduled to) enter
eventually.\footnote{Think eg of an elevator which opens its door and
  doesn't go away. If the scheduler is weakly fair, the process will
  eventually enter.} \footnote{You might want to show that $\Box(\Box
  p\imp\Diamond q) \biimp \Diamond\Box p\imp\Box\Diamond q$.} 
% checked with spin
This is weaker than $\Box\Diamond p\imp\Box\Diamond q$, an example of
which could be: If a process is waiting to enter and the door will
become open inifinitely many times, then the process will (be
scheduled to) enter eventually. \footnote{Think eg of an elevator
  which opens its door, then may go away again, but keeps coming
  back. If the scheduler is strongly fair, the process will eventully
  enter.} \footnote{You might want to show that $(\Box\Diamond
  p\imp\Box\Diamond q) \biimp \Box(\Box\Diamond p \imp \Diamond q)$ or
  also $(\neg\Diamond\Box p\imp\Box\Diamond q) \biimp
  \Box(\neg\Diamond\Box p \imp \Diamond q)$.}


%\input{sr-logic-ml}%Transition systems, modal logic
\subsection{Transition Systems}

A \emph{transition system} $(X,(R_i)_{i\in I},v)$, or \emph{Kripke
  model}, consists of
\begin{itemize}
\item a set $X$ of 'states', also called the carrier of the model,
\item a collection of relations $(R_i)_{i\in I}\subseteq X\times X$,
\item a valuation $v:X\times\bbP\to\bbtwo$ of atomic propositions $\bbP$.
\end{itemize}

\noindent Read $xR_iy$ as $y$ is an $i$-successor of $x$.

\medskip\noindent Why do we have many relations $R_i$? Think of the elements
$i\in I$ as actions, or as the letters of the input alphabet of an
automaton: then in a given state $x$, for each action $i\in I$, we can
have different successors $y$, ie $y$ such that $xR_i y$.

\medskip\noindent To give semantics to distributed processes, we will only need one relation and then we write simply
$(X,R,v)$. 



%\section{Predicate Logic}


This appendix contains additional material. It would fit in after
propositional logic and before temporal logic, for reasons explained
below.


\subsection{Introduction}

Propositional logic is appropriate to specify situations dealing with
finite data, see eg the traffic light example from the lectures. If
one wants to add dynamic features, a good choice is often to add to
the propositional operators (as eg $\neg,\et,\ldots$) temporal
operators, which allow to specify temporal behaviour (as eg always,
sometimes, until).

\medskip\noindent In this section, we briefly look at an even more
powerful extension, namely extending propositional logic by
quantifiers (``for all'', ``there exists''). This allows us to talk
about infinite structures, including data structures such as integers,
lists, etc but also structures like time and hence about dynamic
behaviour.

\medskip\noindent Predicate logic is powerful enough to encode (more
or less) everything that one ever might want to, but this
expressiveness comes at a price: Contrary to propositional logic,
predicate logic is not decidable (ie, there is no algorithm that takes
as input an arbitrary formula $\phi$ and decides whether $\models\phi$
or $\notmodels\phi$; any attempt at writing such an algorithm would
run into the same difficulties as we encountered with Turing's halting
problem, that is, such an algorithm would not be able to
halt on all inputs). Nevertheless, predicate logic is at the basis of
many important formalisms used for the specification and verification
of programs and protocols and so is well worth knowing. Moreover,
temporal logics such as LTL can be understood as a sophisticated way
of limiting the expressive power of predicate logic just enough in
order to make it decidable. 

\subsection{Definitions}

A  \emph{first-order language} $\cal L$ is specified by
\begin{enumerate}
\item a set $\cal F$ of \emph{function symbols} and a natural number
  (called \emph{the arity of $f$}) for each $f\in \cal F$ (function
  symbols of arity 0 are called \emph{constants}),
\item a set $\cal P$ of \emph{predicate symbols} and a natural number
  (called \emph{the arity of $p$}) for each $p\in \cal P$,
\item a set $\mathit{Var}$ of variables.
\end{enumerate}

\noindent
A \emph{term} is either a variable or of the form $f(t_1,\ldots,t_n)$
where $f$ is a function symbol of arity $n$ and the $t_i$ are terms.
An \emph{atom}, or \emph{atomic formula}, is of the form
$p(t_1,\ldots,t_n)$ where $p$ is a predicate symbol of arity $n$ and
the $t_i$ are terms or it is of the form $t_1=t_2$. \emph{Formulae}
are now described by
%
$$\phi ::= p \mid  \neg\phi \mid \phi\et\phi \mid
\forall x.\phi\mid\exists x.\phi$$
where $p$ ranges over atoms and $x$
over variables.

\bigskip\noindent A \emph{model} $M$ (for first-order predicate logic
(FOL)) consists of a non-empty set $A$ and
\begin{enumerate}
\item a function $f^M:A^n\to A$ for each
$n$-ary function symbol\footnote{An `$n$-ary function symbol' is a
  `function symbol of arity $n$'.}  $f$, 
\item a predicate (or relation) $p^M: A^n\to\bbtwo$ for each $n$-ary
  predicate symbol $p$.\footnote{Note that functions $A^n\to\bbtwo$ are in
    one-to-one correspondence to subsets of $A^n$.}
\end{enumerate}

\medskip\noindent
An \emph{environment} (or memory or look-up table or valuation or
interpretation) is
\begin{enumerate}\setcounter{enumi}{2}
\item a function $l:\mathit{Var}\to A$ from variables to
$A$.
\end{enumerate}
The environment $l$ where $x$ has been updated to $a$ is denoted by
$l[x\mapsto a]$. 

\bigskip\noindent
The semantics $\sem{\phi}_{M,l}$ of $\phi$ in a model $M$ under
environment $l$ is now defined as follows ($\sem{\phi}$ plays now the
role of $v(\phi)$ in PL.
%
\begin{itemize}
\item $\sem{x}_{M,l}=l(x)$,
\item $\sem{f(t_1,\ldots t_n)}_{M,l}=f^M(\sem{t_1}_{M,l},\ldots
  \sem{t_n}_{M,l})$,
\item $\sem{p(t_1,\ldots t_n)}_{M,l}=p^M(\sem{t_1}_{M,l},\ldots
  \sem{t_n}_{M,l})$,
\item $\sem{\neg\phi}_{M,l}=\neg\sem{\phi}_{M,l}$,
\item $\sem{\phi\et\psi}_{M,l}=\sem{\phi}_{M,l}\et\sem{\phi}_{M,l}$,
\item $\sem{\forall x.\phi}_{M,l}=1$ if $\sem{\phi}_{M,l[x\mapsto
    a]}=1$ for all $a\in A$,
\item $\sem{\exists x. \phi}_{M,l}=1$ if $\sem{\phi}_{M,l[x\mapsto
    a]}=1$ for some $a\in A$.
\end{itemize}

\medskip\noindent
$\sem{\phi}_{M}=1$ if $\sem{\phi}_{M,l}=1$ for all $l$.
$\sem{\phi}=1$ if for $\sem{\phi}_{M}=1$ for all $M$.

\medskip\noindent
We also write 
%
\begin{itemize}
\item $M,l\models\phi$ for $\sem{\phi}_{M,l}=1$,
\item $M\models\phi$ for $\sem{\phi}_{M}=1$,
\item $M\models\Phi$ if $M\models\phi$ for all $\phi\in\Phi$,
\end{itemize}

Validity and consequence are defined as follows.
\begin{itemize}
\item $\models\phi$ (`$\phi$ is valid') if $M\models\phi$ for all $M$,
\item $\Phi\models\phi$ (`$\phi$ is a consequence of the set of
  formulae $\Phi$') if for all $M$ it holds that $M\models\Phi$
  implies $M\models\phi$.
\end{itemize}



%\input{sr-logic-transitionsystems}
\section{Logic for Transition Systems (Modal Logic)}\label{sec:ml}

This section contains additional material. It would fit after the
definition of a transition system. This section shows that temporal
logic is part of a wider field of logic called modal logic. Modal
logics come indifferent shapes and sizes and are used all over
computer science to model very different phenomena such as knowledge,
belief, obligations, etc. 

\medskip\noindent \textbf{Syntax of ML.} $\phi::=p \mid \neg\phi \mid
\phi\et\phi \mid \Box_i\phi \mid \Diamond_i\phi$

\bigskip\noindent\textbf{Semantics of ML. } Let $M=(X,(R_i)_{i\in
  I},v)$ be a Kripke model and $x\in X$.
\begin{itemize}
\item $M,x\models p$ \ if $v(x,p)=1$,
\item $M,x\models \neg\phi$ \ if not $M,x\models \phi$ (notation:
  $M,x\notmodels \phi$),
\item $M,x\models \phi\et\psi$ \ if $M,x\models \phi$ and $M,x\models \phi$,
\item $M,x\models \Box_i\phi$ \ if $M,y\models\phi$ for all $y$ such
  that $x R_i y$,\footnote{Other notations for $x R_i y$ are:
    $(x,y)\in R_i$, $R_i(x,y)$ $R_i(x,y)=1$,
    $x\stackrel{i}{\longrightarrow}y$, $x\longrightarrow_i y$.}
\item $M,x\models \Diamond_i\phi$ \ if $M,y\models\phi$ for some $y$ such
  that $x R_i y$.
\end{itemize}

\noindent $M\models\phi$ \ if $M,x\models\phi$ for all $x$.\\
$M\models\Phi$ \ if $M\models\phi$ for all $\phi\in\Phi$.\\
$\Phi\models\phi$ \  if $M\models\Phi$ implies $M\models\phi$ for all $M$. \footnote{Notation: $\psi\models\phi$ for $\{\psi\}\models\phi$; $\models\phi$ for $\emptyset\models\phi$.} \\
$(X,(R_i)_{i\in I})\models\phi$ \ if $(X,(R_i)_{i\in I},v)\models\phi$
for all $v$ \footnote{Terminology: $(X,(R_i)_{i\in I})$ is called a
  \emph{Kripke frame} in this context.}

\bigskip\noindent\textbf{Notation. } One often writes $[i]$ for
$\Box_i$ and $\langle i\rangle$ for $\Diamond_i$. In many cases, there
will be just one transition relation $R$. We then write $(X,R,v)$ and
the operators as $\Box$, $\Diamond$.

\bigskip\noindent\textbf{Example 1. } Consider a model $(\bbN,\le,v)$,
which has as a carrier the natural numbers (= non-negative integers).
Thinking of $x\le y$ as $y$ is in the future of $x$, we have that
$\Box$ means `always' and $\Diamond$ means `sometimes'.

\bigskip\noindent\textbf{Example 2. } Consider a model $(\bbN,S,v)$,
where $\bbN$ is as before, but $S$ is the relation $\{(x,x+1) \mid
x\in \bbN\}$. Thinking of $\bbN$ again as the flow of time, we have
that $\Box$ means `at the next point in time'. Note that
$\Box\phi\biimp \Diamond\phi$.

\bigskip\noindent\textbf{Example 3. } Consider an arbitrary model
$(X,R,v)$, where we think of $X$ as a set of worlds (or possible
scenarios) and of $xRy$ as $y$ being an alternative to $x$. Then we
can read $\Box$ as `necessarily' and $\Diamond$ as `possibly'. [This
example is the one with which the area of modal logic originated ca
hundred years ago.]

\bigskip\noindent\textbf{Example 4. } Consider a model $(X,R_i,v)$,
where we think of $X$ as a set of worlds and of $xR_iy$ as: The world
$y$ looks the same as $x$ according to the facts known to agent $i$.
Then we can read $\Box_i$ as `agent $i$ knows'. [This example is
important in the specification of multi-agent systems.]




\bigskip\noindent\textbf{Remark. } A Kripke model can be seen as a
model for FOL where the language consists of one binary predicate
symbol for each $i\in I$ and one unary predicate symbol for each
$p\in\bbP$. We then see that the modal language is a restriction of
FOL. For example, formulae have at most one `free' variable (the $x$
in the semantics definition). Moreover, $\Box$ and $\Diamond$ are
restricted quantifiers in that they do not quantify over all elements
of the model but only over the successors of a particular element. One
of the benefits from this restriction is that the logic becomes
decidable (Proof: Adapt the semantic tableau method from propositional
logic (we won't do this in the course, but it is not very difficult)).
There is even a stronger property: Each satisfiable formula is
satisfiable in a finite model (and we have seen in the lectures that
this is not true for FOL).
%Tableaux for LTL
\subsection{Transition System Semantics of LTL, Some Remarks on Spin}
%
Let $M=(X,R,v)$ be a transition system and $x\in X$. We call $x$ the
\emph{initial state} of $M$. For LTL-formulae $\phi$, we define
\[M,x\models\phi \textrm{ \ if \ } \sigma,0\models\phi \textrm{ for
  all sequences $\sigma$ through $M$ starting at $x$.}\]


\bigskip\noindent\textbf{Remark. }  This is the semantics used by the
SPIN model checker: A Promela program defines a transition system $M$
together with an initial state $x$; checking a formula $\phi$ against
a Promela model is the same as asking the question whether
\[M,x\models\phi.\]
In case that $M,x\notmodels\phi$, the model checker will produce a
counter example to $\phi$, that is, a particular sequence $\sigma$
such that \[\sigma,0 \notmodels\phi,\] or, which is the
same, \[\sigma,0 \models\neg\phi,\]

\smallskip\noindent Note that, to check $\phi$ using Spin, one has to
use the negation \texttt{!($\phi$)} as in \texttt{spin -f
  '!($\phi$)'}. Spins tries to find a counterexample to $\phi$ by
finding an example of a sequence satisfying $\neg\phi$.

\bigskip\noindent\textbf{Remark on Interleaving Semantics.} The way
SPIN assigns a transition system to a Promela program is an example of
interleaving semantics: No two independent processes are allowed to
move at the same time.  This assumption sounds
unrealistic, but it doesn't matter.

\bigskip\noindent\textbf{Semantics of $U$ (Spin's until). } Spin knows
the temporal operators from Section~\ref{sec:LTL:def}, but uses a
variation of $\until$ written as $U$. Its semantics is given by
\begin{itemize}
\item $\sigma,n\models p U q$ \ if \ \
  \parbox{15em}{$\sigma,m\models q$ for some $m\ge n$ and\\
    $\sigma,k\models p$ for all $k$ with $n\le k < m$.}
\end{itemize}
$U$ can be expressed using $\until$: $p U q \biimp q \ou (p \et (p
\until q))$ but not vice versa: formulae built from $U$ are stutter
invariant (see below). Accordingly, $\bot U q$ does not express $\next q$ but we
have the laws $\bot U q \biimp q$ and $q\imp(p U q)$.

\medskip\noindent $\until$ can be expressed as $(p\until q) \biimp
\next(p U q)$.\footnote{In Spin-LTL one writes \texttt{X} for
  $\next$.}

\bigskip\noindent\textbf{Stutter invariance } An LTL-formula $\phi$ is
said to be stutter-invariant if $(v_n)_{n_\bbN}\models\phi$ implies
that $(v'_n)_{n\in\bbN}\models\phi$ for all stuttervariants
$(v'_n)_{n\in\bbN}$ of $(v_n)_{n\in\bbN}$, where $(v'_n)_{n\in\bbN}$ is
called a \emph{stuttervariant} of $(v_n)_{n\in\bbN}$ if there is a
surjective function $f:\bbN\to\bbN$ such that $f(0)=0$ and either
$f(n+1)=f(n)$ or $f(n+1)=f(n)$ and $v'_n=v_{f(n)}$.


\subsection{Semantic Tableaux for LTL}

\subsubsection*{Motivation and Explanation}

LTL-tableaux are an extension of propositional tableau. Although
technically more complicated, the basic ideas are not
difficult. First,  we need some preliminaries.

\medskip\noindent Recall that, in order to keep the number of rules
small, we used certain laws to transform formulae to a form making the
rules applicable. These laws are summarised in
Table~\ref{table:prop-laws}.  Remember that we apply them from left to
right in a tableau.

\begin{table}
\fbox{\parbox{\textwidth}{
\begin{align*}
(\phi\biimp\psi) & \biimp  (\phi\imp \psi)\et(\psi\imp \phi)\\
\neg(\phi\ou\psi)& \biimp \neg\phi\et\neg\psi\\
\neg(\phi\et\psi)& \biimp \neg\phi\ou\neg\psi\\
\phi\imp\psi &\biimp \neg\phi\ou\psi\\
\neg(\phi\imp\psi)&\biimp \phi\et\neg\psi
\end{align*}
}}
\caption{Propositional laws}\label{table:prop-laws}
\end{table}

\medskip\noindent To deal with the new temporal operators, we add the
laws of Table~\ref{table:ltl-laws}. (Exercise: show that these laws
are valid.) They are again to be applied from left to right. Their
effect is that in the leaves of a propositional tableau, there are
only the following kind of formulae: atomic propositions, negations of
atomic propositions, formulae of the kind $\next\phi$.\footnote{We may
  want to add, for the Spin-version of until,
\begin{align*}
  \phi U\psi & \biimp \psi \ou (\phi\et\next(\phi U\psi)))\\
  \neg(\phi U\psi) & \biimp (\neg\psi \et (\neg\phi\ou
  \next\neg(\phi U\psi)))
\end{align*}
}

%
\begin{table}
  \fbox{\parbox{\textwidth}{
      \begin{align*}
        \neg\next\phi & \biimp \next\neg\phi \\
        \neg\Diamond\phi & \biimp \Box\neg\phi \\
        \neg\Box\phi & \biimp \Diamond\neg\phi\\
% \end{align*}
% and
% \begin{align*}
        \Diamond\phi & \biimp \phi\ou\next\Diamond\phi \\
        \Box\phi & \biimp \phi\et\next\Box\phi\\
        \phi\until\psi & \biimp \next(\psi \ou (\phi\et(\phi\until\psi)))\\
        \neg(\phi\until\psi) & \biimp \next(\neg\psi \et(\neg\phi\ou
        \neg(\phi\until\psi)))
\end{align*}
}}
\caption{Temporal laws}\label{table:ltl-laws}
\end{table}


\paragraph{How do we build an LTL-tableau? }
%
Start building a propositional tableau applying the laws from
Tables~\ref{table:prop-laws} and \ref{table:ltl-laws}.  Now, all the
leaves of this tableau contain only atomic propositions or negations
of atomic propositions or formulae whose outermost connective is
$\next$.

\medskip\noindent These nodes are leaves in a propositional tableaux.
No further propositional reasoning is possible. The leaves correspond
to the states of LTL-models.

\medskip\noindent How to make a transition from one state to the next
will be described now. We first need two definitions.

\medskip\noindent\textbf{Definition. } Let us agree to call a node
labelled with a set of formulas $\Phi$ a
\textbf{state},\footnote{X-node in \cite{benari:logic}} if $\Phi$
contains only atomic propositions or negations of atomic propositions
or formulae whose outermost connective is $\next$; moreover, we assume
that there are no contradictions among the (negated) atomic
propositions.

\medskip\noindent\textbf{Definition. } For a state labelled $\Phi$
define $\Phi'=\{\phi \mid \next\phi\in\Phi\}$.

\medskip\noindent In other words, $\Phi'$ erases the outermost $\next$
operator from formulas in $\Phi$.

\medskip\noindent We now come to the rule for $\next$. The notation
$\Phi'$ has been devised in a way such that, if \emph{now} $\Phi$
holds then at the \emph{next} time-point $\Phi'$ must hold \ldots if
that is not clear, pause to think about it.  We therefore would like
to say that any state $\Phi$ has precisely one child, namely $\Phi'$.
Unfortunately, this does not work, since the tableau for, eg, $\Box p$
would be infinite (easy exercise!).
%
We solve this problem as follows. Given a state labelled $\Phi$, if
there is already a node labelled $\Phi'$, 
then we do not create a child of $\Phi$ and instead ``loop back'' to
the already existing node labelled by $\Phi'$. Otherwise create a child labelled $\Phi'$ and continue as
above.

\medskip\noindent\textbf{Definition. } A tableau is complete if there
is no further rule to apply. A tableau is closed if all branches
contain a contradiction, otherwise it is open.

\medskip\noindent\textbf{Definition. } Given a complete, open tableau,
we extract a graph $(V,E)$ from where the vertices $V$ are the states
of the tableau and there is an edge $(s,t)\in E$ whenever there is
path $(s,s_1,\ldots, s_n,t)$ in the tableau such that the $s_i$ are
non-state nodes. We also define a mapping $F$ from vertices to
formulas. If $(\Phi,\Psi_1,\ldots, \Psi_n,\Phi_2)$ are the set of
formulas labelling $(s,s_1,\ldots, s_n,t)$ in the tableau, then $F(t)$
is defined as $\bigcup\{\Psi_i\mid 1\le i \le
n\}\cup\Phi_2$.

\medskip\noindent\textbf{Extracting a model from a tableau. }  If the
graph of a complete open tableau has a vertex that has no successor,
then the path leading to that vertex is a model of the formula at the
root of the tableau and the formula is therefore satisfiable. If the
graph has a cycle, this cycle is is model if it is selfulfilling,
defined as follows.

\medskip\noindent\textbf{Definition. }  A cycle is
\textbf{self-fulfilling} if
\begin{enumerate}
\item for all vertices $s$ in the cycle and all $\Diamond\phi\in F(s)$
  there is some vertex $t$ in the cycle such that $\phi\in F(t)$
\item for all vertices $s$ in the cycle and all $(\phi\until\psi)\in
  F(s)$ there is some node $t$ in the cycle such that $\psi \in F(t)$
\end{enumerate}
Intuitively, we think of a formula $\Diamond\phi$ as a \emph{promise}
to make $\phi$ true. Now, recall that building a tableau we are trying
to construct a sequence satisfying the formulas in the tableau. In
order to satisfy the promise $\Diamond\phi$, we need that $\phi$ is
true in some node of the sequence. If this happens, we call the
sequence self-fulfilling.

\medskip\noindent\textbf{A summary} of the construction of an
LTL-tableau is given in Table~\ref{table:ltl-tableau}. If the tableau
is not closed, we extract a graph from the tableau. Then the formula
at the root of the tableau is satisfiable if the graph contains a
self-fulfilling cycle.

\begin{table}
\fbox{\parbox{\textwidth}{
\begin{enumerate}
\item Build a complete propositional tableau for $\Psi$ using the laws of
  Tables~\ref{table:prop-laws} and \ref{table:ltl-laws}.
\item We continue as follows (with $\Phi$ ranging over the leaves of
  the propositional tableau just built):

\begin{itemize}
\item If $\Phi$ contains a contradiction, then $\Phi$ has no child and
  the branch is closed.
\item If $\Phi$ does not contain a contradiction:
\begin{itemize}
\item If $\Phi'$ appears on the path from the root to $\Phi$, then no
  new child of $\Phi$ is created but one loops back to $\Phi'$.
\item If $\Phi'$ is empty, then $\Phi$ has no child.
\item Otherwise: $\Phi$ has a child $\Phi'$. (The notation $\Phi'$ was
  defined above.) Go back to 1., continuing
  to build a propositional tableau for $\Psi=\Phi'$.
\end{itemize}
\end{itemize}
\end{enumerate}
}}
\caption{How to build an LTL-tableau with root $\Psi$}\label{table:ltl-tableau}
\end{table}



% \subsubsection{Semantic Tableaux for LTL}\label{sem-tab-LTL}

% The definition of a complete semantic tableau is given in
% Table~\ref{table:complete}. It is closed or open as explained in the
% following.
% %
% \begin{table}
% \fbox{\parbox{\textwidth}{
%   In a complete semantic LTL-tableau, each
%   node is either a state or not. A non-state node satisfies
%   one of the following conditions (which are exactly the same as for
%   propositional tableaux).

% \begin{itemize}
% \item $\Phi$ contains a formula $\phi \ou\psi$ and has exactly
%   two children given by $(\Phi\cup\{\phi\})\setminus\{\phi \ou\psi\}$
%   and $(\Phi\cup\{\psi\})\setminus\{\phi \ou\psi\}$.
    
% \item $\Phi$ contains a formula $\phi \et\psi$ and has exactly
%   one child given by $(\Phi\cup\{\phi,\psi\})\setminus\{\phi
%   \et\psi\}$.
      
% \item $\Phi$ contains a formula $\neg\neg\phi$ and has exactly one
%   child given by $(\Phi\cup\{\phi\})\setminus\{\neg\neg\phi\}$.
% \end{itemize}


% \noindent For state nodes we have:
% \begin{itemize}
% \item State nodes $\Phi$ that contain a contradiction, or
%   have a subsuming ancestor or have empty $\Phi'$ have no child.
%   Otherwise $\Phi$ has one child $\Phi'$.
% \end{itemize}
% }
% }
% \caption{Complete semantic tableau for LTL}\label{table:complete}
% \end{table}

% Note how the first three items of Table~\ref{table:complete}
% correspond to those in
% Tables~\ref{table:tableau-rules}~\ref{table:tableau-rules-short}.  The
% last item of Table~\ref{table:complete} can be written in the notation
% of Table~\ref{table:ltl-rules-short}.

% \begin{table}
% \fbox{\parbox{\textwidth}{
% \[
% \rules{p_1,p_2,\ldots \neg q_1,\neg q_2,\ldots,\next\phi_1,\next\phi_2,\ldots }{\phi_1,\phi_2,\ldots}
% \]
% }}
% \caption{LTL tableau rule for state nodes (``steps in time'')}\label{table:ltl-rules-short}
% \end{table}

% \medskip The way we defined `state', all leaves are
% states.  If a leaf contains a contradiction it is called closed.
% If a leaf does not contain a contradiction and does not have a
% subsuming ancestor it is open. If a leaf does have a subsuming
% ancestor then it is open or closed depending on whether the loop is
% self-fulfilling or not. In other words:

% \medskip\noindent A complete semantic tableau for $\{\neg\phi\}$ is
% \emph{open} if it contains an \emph{open leaf} $\Phi$, that is, a leaf
% $\Phi$ that contains no contradiction and, moreover, if $\Phi$ is in a
% loop, then the loop associated with $\Phi$ is self-fulfilling.


% \medskip\noindent The path from the root to the leaf gives rise to an
% (infinite) sequence satisfying $\{\neg\phi\}$. A tableau is
% \emph{closed} if it is not open. A formula $\phi$ is valid if there is
% some closed semantic tableau which has $\{\neg\phi\}$ as its root.


%%% Local Variables: 
%%% mode: latex
%%% TeX-master: "sr-logic"
%%% End: 
% Temporal logic
\section{LTL Model Checking with Semantic Tableaux}

To check whether a property $\phi$ holds in a Kripke model $M$, we
`synchronise' the semantic tableaux method for the validity of $\phi$
with the transition system $M$.

\medskip\noindent
Let $M=(X,R,v)$ where $v:X\times\bbP\to\bbtwo$ for a fixed set $\bbP$ of
atomic propositions. Also fix an initial state $x_0\in X$. 
%
We want to build a tableau for $\phi$ and $M$. The nodes $(x,\Phi)$ of
the tableau consist of states $x\in X$ and sets of LTL-formulae
$\Phi$. 

% A node $(x,\Phi)$ is called propositional or transitional if
% $\Phi$ is so called. 

\medskip\noindent The root is $(x_0,\{\neg\phi\}\cup \{p\mid
v(x_0)(p)=1\}\cup\{\neg p \mid v(x_0)(p)=0\} )$.

\medskip\noindent For a non-state node $\Phi$, the children of
$(x,\Phi)$ are $(x,\Psi)$ where $\Psi$ is a child of $\Phi$ according
to the three rules for propositional nodes from above.

\medskip\noindent For state nodes $\Phi$, $(x,\Phi)$ is a leaf (ie, it
has no child) if $\Phi$ contains a contradiction or if $\Phi'$ is
empty or if the successors $(x',\Phi')$ of $(x,\Phi)$ appear on a path
from the root to $(x,\Phi)$. 
%
Otherwise the children of $(x,\Phi)$ are
%
$$(y,\Phi'\cup\{p\in\bbP \mid
v(y,p)=1\}\cup \{\neg p\in\bbP \mid v(y,p)=0\})$$ 
%
for all successors $y$ of $x$, that is, for all $y$ such that $xRy$.


% \medskip\noindent In analogy to the previous section, we say that if
% $\Phi$ has a subsuming ancestor $\Gamma$, then the path from
% $(x,\Gamma)$ to $(x,\Phi)$ is the loop associated with
% $(x,\Phi)$.

% \medskip As before, a leaf $(x,\Phi)$ is called \emph{open} (a) if
% $\Phi$ does not contain a contraction or (b) if there is a loop
% associated with $(x,\Phi)$ that is self-fulfilling. A leaf is called
% \emph{closed} if it is not open.

% \medskip\noindent $M,x_0\models\phi$ if there is a complete tableau
% with root $(x_0,\{\neg\phi\})$ that has no open leaf.  Conversely, if
% there is a complete tableau with root $(x_0,\{\neg\phi\})$ that has an
% open leaf, then the path from the root to that leaf (possibly extended
% by the associated loop) is a counter-example to $\phi$. In Spin, the
% associated loop is called \emph{acceptance cycle}.

%%% Local Variables: 
%%% mode: latex
%%% TeX-master: "sr-logic"
%%% End: 
% 

\bibliography{sr-logic}
\bibliographystyle{abbrv}





\newpage
\appendix
\section{Predicate Logic}


This appendix contains additional material. It would fit in after
propositional logic and before temporal logic, for reasons explained
below.


\subsection{Introduction}

Propositional logic is appropriate to specify situations dealing with
finite data, see eg the traffic light example from the lectures. If
one wants to add dynamic features, a good choice is often to add to
the propositional operators (as eg $\neg,\et,\ldots$) temporal
operators, which allow to specify temporal behaviour (as eg always,
sometimes, until).

\medskip\noindent In this section, we briefly look at an even more
powerful extension, namely extending propositional logic by
quantifiers (``for all'', ``there exists''). This allows us to talk
about infinite structures, including data structures such as integers,
lists, etc but also structures like time and hence about dynamic
behaviour.

\medskip\noindent Predicate logic is powerful enough to encode (more
or less) everything that one ever might want to, but this
expressiveness comes at a price: Contrary to propositional logic,
predicate logic is not decidable (ie, there is no algorithm that takes
as input an arbitrary formula $\phi$ and decides whether $\models\phi$
or $\notmodels\phi$; any attempt at writing such an algorithm would
run into the same difficulties as we encountered with Turing's halting
problem, that is, such an algorithm would not be able to
halt on all inputs). Nevertheless, predicate logic is at the basis of
many important formalisms used for the specification and verification
of programs and protocols and so is well worth knowing. Moreover,
temporal logics such as LTL can be understood as a sophisticated way
of limiting the expressive power of predicate logic just enough in
order to make it decidable. 

\subsection{Definitions}

A  \emph{first-order language} $\cal L$ is specified by
\begin{enumerate}
\item a set $\cal F$ of \emph{function symbols} and a natural number
  (called \emph{the arity of $f$}) for each $f\in \cal F$ (function
  symbols of arity 0 are called \emph{constants}),
\item a set $\cal P$ of \emph{predicate symbols} and a natural number
  (called \emph{the arity of $p$}) for each $p\in \cal P$,
\item a set $\mathit{Var}$ of variables.
\end{enumerate}

\noindent
A \emph{term} is either a variable or of the form $f(t_1,\ldots,t_n)$
where $f$ is a function symbol of arity $n$ and the $t_i$ are terms.
An \emph{atom}, or \emph{atomic formula}, is of the form
$p(t_1,\ldots,t_n)$ where $p$ is a predicate symbol of arity $n$ and
the $t_i$ are terms or it is of the form $t_1=t_2$. \emph{Formulae}
are now described by
%
$$\phi ::= p \mid  \neg\phi \mid \phi\et\phi \mid
\forall x.\phi\mid\exists x.\phi$$
where $p$ ranges over atoms and $x$
over variables.

\bigskip\noindent A \emph{model} $M$ (for first-order predicate logic
(FOL)) consists of a non-empty set $A$ and
\begin{enumerate}
\item a function $f^M:A^n\to A$ for each
$n$-ary function symbol\footnote{An `$n$-ary function symbol' is a
  `function symbol of arity $n$'.}  $f$, 
\item a predicate (or relation) $p^M: A^n\to\bbtwo$ for each $n$-ary
  predicate symbol $p$.\footnote{Note that functions $A^n\to\bbtwo$ are in
    one-to-one correspondence to subsets of $A^n$.}
\end{enumerate}

\medskip\noindent
An \emph{environment} (or memory or look-up table or valuation or
interpretation) is
\begin{enumerate}\setcounter{enumi}{2}
\item a function $l:\mathit{Var}\to A$ from variables to
$A$.
\end{enumerate}
The environment $l$ where $x$ has been updated to $a$ is denoted by
$l[x\mapsto a]$. 

\bigskip\noindent
The semantics $\sem{\phi}_{M,l}$ of $\phi$ in a model $M$ under
environment $l$ is now defined as follows ($\sem{\phi}$ plays now the
role of $v(\phi)$ in PL.
%
\begin{itemize}
\item $\sem{x}_{M,l}=l(x)$,
\item $\sem{f(t_1,\ldots t_n)}_{M,l}=f^M(\sem{t_1}_{M,l},\ldots
  \sem{t_n}_{M,l})$,
\item $\sem{p(t_1,\ldots t_n)}_{M,l}=p^M(\sem{t_1}_{M,l},\ldots
  \sem{t_n}_{M,l})$,
\item $\sem{\neg\phi}_{M,l}=\neg\sem{\phi}_{M,l}$,
\item $\sem{\phi\et\psi}_{M,l}=\sem{\phi}_{M,l}\et\sem{\phi}_{M,l}$,
\item $\sem{\forall x.\phi}_{M,l}=1$ if $\sem{\phi}_{M,l[x\mapsto
    a]}=1$ for all $a\in A$,
\item $\sem{\exists x. \phi}_{M,l}=1$ if $\sem{\phi}_{M,l[x\mapsto
    a]}=1$ for some $a\in A$.
\end{itemize}

\medskip\noindent
$\sem{\phi}_{M}=1$ if $\sem{\phi}_{M,l}=1$ for all $l$.
$\sem{\phi}=1$ if for $\sem{\phi}_{M}=1$ for all $M$.

\medskip\noindent
We also write 
%
\begin{itemize}
\item $M,l\models\phi$ for $\sem{\phi}_{M,l}=1$,
\item $M\models\phi$ for $\sem{\phi}_{M}=1$,
\item $M\models\Phi$ if $M\models\phi$ for all $\phi\in\Phi$,
\end{itemize}

Validity and consequence are defined as follows.
\begin{itemize}
\item $\models\phi$ (`$\phi$ is valid') if $M\models\phi$ for all $M$,
\item $\Phi\models\phi$ (`$\phi$ is a consequence of the set of
  formulae $\Phi$') if for all $M$ it holds that $M\models\Phi$
  implies $M\models\phi$.
\end{itemize}



%\input{sr-logic-transitionsystems}
\section{Logic for Transition Systems (Modal Logic)}\label{sec:ml}

This section contains additional material. It would fit after the
definition of a transition system. This section shows that temporal
logic is part of a wider field of logic called modal logic. Modal
logics come indifferent shapes and sizes and are used all over
computer science to model very different phenomena such as knowledge,
belief, obligations, etc. 

\medskip\noindent \textbf{Syntax of ML.} $\phi::=p \mid \neg\phi \mid
\phi\et\phi \mid \Box_i\phi \mid \Diamond_i\phi$

\bigskip\noindent\textbf{Semantics of ML. } Let $M=(X,(R_i)_{i\in
  I},v)$ be a Kripke model and $x\in X$.
\begin{itemize}
\item $M,x\models p$ \ if $v(x,p)=1$,
\item $M,x\models \neg\phi$ \ if not $M,x\models \phi$ (notation:
  $M,x\notmodels \phi$),
\item $M,x\models \phi\et\psi$ \ if $M,x\models \phi$ and $M,x\models \phi$,
\item $M,x\models \Box_i\phi$ \ if $M,y\models\phi$ for all $y$ such
  that $x R_i y$,\footnote{Other notations for $x R_i y$ are:
    $(x,y)\in R_i$, $R_i(x,y)$ $R_i(x,y)=1$,
    $x\stackrel{i}{\longrightarrow}y$, $x\longrightarrow_i y$.}
\item $M,x\models \Diamond_i\phi$ \ if $M,y\models\phi$ for some $y$ such
  that $x R_i y$.
\end{itemize}

\noindent $M\models\phi$ \ if $M,x\models\phi$ for all $x$.\\
$M\models\Phi$ \ if $M\models\phi$ for all $\phi\in\Phi$.\\
$\Phi\models\phi$ \  if $M\models\Phi$ implies $M\models\phi$ for all $M$. \footnote{Notation: $\psi\models\phi$ for $\{\psi\}\models\phi$; $\models\phi$ for $\emptyset\models\phi$.} \\
$(X,(R_i)_{i\in I})\models\phi$ \ if $(X,(R_i)_{i\in I},v)\models\phi$
for all $v$ \footnote{Terminology: $(X,(R_i)_{i\in I})$ is called a
  \emph{Kripke frame} in this context.}

\bigskip\noindent\textbf{Notation. } One often writes $[i]$ for
$\Box_i$ and $\langle i\rangle$ for $\Diamond_i$. In many cases, there
will be just one transition relation $R$. We then write $(X,R,v)$ and
the operators as $\Box$, $\Diamond$.

\bigskip\noindent\textbf{Example 1. } Consider a model $(\bbN,\le,v)$,
which has as a carrier the natural numbers (= non-negative integers).
Thinking of $x\le y$ as $y$ is in the future of $x$, we have that
$\Box$ means `always' and $\Diamond$ means `sometimes'.

\bigskip\noindent\textbf{Example 2. } Consider a model $(\bbN,S,v)$,
where $\bbN$ is as before, but $S$ is the relation $\{(x,x+1) \mid
x\in \bbN\}$. Thinking of $\bbN$ again as the flow of time, we have
that $\Box$ means `at the next point in time'. Note that
$\Box\phi\biimp \Diamond\phi$.

\bigskip\noindent\textbf{Example 3. } Consider an arbitrary model
$(X,R,v)$, where we think of $X$ as a set of worlds (or possible
scenarios) and of $xRy$ as $y$ being an alternative to $x$. Then we
can read $\Box$ as `necessarily' and $\Diamond$ as `possibly'. [This
example is the one with which the area of modal logic originated ca
hundred years ago.]

\bigskip\noindent\textbf{Example 4. } Consider a model $(X,R_i,v)$,
where we think of $X$ as a set of worlds and of $xR_iy$ as: The world
$y$ looks the same as $x$ according to the facts known to agent $i$.
Then we can read $\Box_i$ as `agent $i$ knows'. [This example is
important in the specification of multi-agent systems.]




\bigskip\noindent\textbf{Remark. } A Kripke model can be seen as a
model for FOL where the language consists of one binary predicate
symbol for each $i\in I$ and one unary predicate symbol for each
$p\in\bbP$. We then see that the modal language is a restriction of
FOL. For example, formulae have at most one `free' variable (the $x$
in the semantics definition). Moreover, $\Box$ and $\Diamond$ are
restricted quantifiers in that they do not quantify over all elements
of the model but only over the successors of a particular element. One
of the benefits from this restriction is that the logic becomes
decidable (Proof: Adapt the semantic tableau method from propositional
logic (we won't do this in the course, but it is not very difficult)).
There is even a stronger property: Each satisfiable formula is
satisfiable in a finite model (and we have seen in the lectures that
this is not true for FOL).

\section{Questions and Exercises}
Questions marked with * relate to material relevant for the
exam. Questions of particular importance for the exam are marked with
**.
\begin{question}* \ 
  A traffic light can be modelled in propositional logic by three
  atomic propositions red, yellow, green. Without further
  specification, this means that a traffic light can be in $2^3=8$
  different states. Now suppose you want to model a crossing with 4
  traffic lights (for cars) and 4 traffic lights for cyclists (the
  latter have only red and green). How many possible states are there
  for the crossing now? What if you add 8 more traffic lights (two
  colours each) for pedestrians? Explain the notion of `state
  explosion'.
\end{question}

\begin{question}* \
For each of the following formulae determine whether they are valid
and whether they are satisfiable. 
\begin{center}
\begin{tabular}{c}
$p\imp p$\\
$p\et\neg p$\\
$p\et\neg q$\\
$p\ou\neg p$\\
$p$\\
$\neg p$
\end{tabular}
\end{center}

\end{question}

\begin{question}* \
  List all models of the following traffic light specification
  ($r,y,g$ for red, yellow, green).
\[
(r\ou y)\imp \neg g
\]
Do this using a truth table and do this again using a semantic
tableau. Compare the two procedures.  Add a formula that makes sure
that not all colours can be dark simultaneously.
\end{question}
 
\begin{question}
\begin{enumerate}
\item
  Explain in what sense propositional logic can also be defined by the
  smaller grammar
%
$$\phi ::= p \mid  \neg\phi \mid \phi\et\phi$$
%
 [Hint: Use the equivalences of Example~\ref{exle:prop:valid}.]
\item Similarly, show that from each of the following sets of
  connectives $\{\bot,\imp\}$, $\{\neg,\ou\}$ all other connectives
  can be defined. 
\end{enumerate}
\end{question}

\begin{question}** \
Use semantic tableaux to check whether the following formulae are valid
and, in case they are not, give a counterexample.
\begin{center}
\begin{tabular}{c}
$((p\et q)\imp r) \imp (p\imp (q\imp r))$\\ %v
$(p\imp (q\imp r))\imp ((p\et q)\imp r)$\\  %v
$((p\et q)\imp r) \imp ((p\imp q)\imp r)$\\ %nv
$((p\imp q)\imp r)\imp ((p\et q)\imp r)$\\  %v
%$(p\imp (q\ou\neg r))\et p\et r \imp q$\\
$(p\et q \et (q\imp( r\imp p)))\imp r$\\ %nv
%$(p \et \neg(p\et q) \et (\neg r\imp q) \et (r\imp q)) \imp s$\\
$(r\et q) \imp (((r\imp s)\et q) \ou \neg(q\imp s))$ \\ %(0607)
$(p\imp (q\imp r))\imp((p\imp q)\imp (p\imp r))$ \\%(0708)
%some new ones:
% $(p\imp(p\et r))\ou(r\imp(p\et\neg p))$\\%valid
% $(p\ou q \imp r )\ou((r\imp p)\et(r\imp q))$\\%valid
% $(p\imp(q\ou r))\ou\neg(\neg q\imp r)$\\%valid
% $(p\et q)\ou(r\et q)\ou\neg q\ou (\neg p \et q \et r)$\\%valid
% $\neg q\imp(\neg(p\et q)\imp(r\et q))$\\%not valid  
\end{tabular}
\end{center}
\end{question}


\begin{question} \ \label{qn:traffic-light-tableau} 
\begin{enumerate}
\item
Show that the following FOL-formulae are valid.
\begin{center}
\begin{tabular}{c}
$\forall x.\phi \imp \exists x.\phi$\\
$\exists y.\forall x.\phi\imp \forall x.\exists y. \phi$\\
$\forall x.(\phi\imp \psi) \imp(\forall x. \phi \imp \forall x. \psi)$\\
\end{tabular}
\end{center}
\item
Give counter-examples for 
\begin{center}
\begin{tabular}{c}
$\forall x.\exists y. \phi \imp \exists y.\forall x.\phi$\\
$(\forall x. \phi \imp \forall x. \psi) \imp \forall x. (\phi\imp \psi) $\\
\end{tabular}
\end{center}
\end{enumerate}
\end{question}

\begin{question}
  Why is PL decidable. Why does this argument not apply to FOL
  (First-order predicate logic)? Is FOL decidable?
\end{question}


\begin{question}* \
Define 
\begin{itemize}
\item $M,n\models \phi\atnext\psi$ \ if \ \
  \parbox{20em}{$M,m\notmodels\psi$ for all $m> n$ or \\
    $M,k\models\phi$ for the smallest $k> n$ with $M,k\models\psi$.}
\item $M,n\models \phi\while\psi$ \ if \ \
  \parbox{20em}{$M,m\notmodels\psi$ for some $m > n$ and\\
    $M,k\models\phi\et\psi$ for all $k$ with $n< k< m$.}
\item $M,n\models \phi\before\psi$ \ if \ \
  \parbox{20em}{for all $m> n$ with $M,m\models\psi$\\
                there is  $k$ with $n< k< m$ such that $M,k\models\phi$.}
\end{itemize}
Express these operators using LTL.
\end{question}

\begin{question}** \
  For each of the following LTL-formulae, either show that the formula
  is valid or give a counterexample.
\begin{enumerate}
\item $\next\Box p \imp \Box\next p$,
\item $\Box(p\ou q) \imp \Box p \ou \Box q$,
\item $\Box p \ou \Box q \imp \Box(p\ou q)$,
\item $\Box\Diamond\phi\imp\Diamond\Box\phi$,
\item $\Box(\phi\imp\next\Diamond\psi) \et \Diamond \phi
  \imp\Diamond\psi$.
\item $\Box q \et (p\until q) \imp \Box p$,
\item $\next\phi \biimp \bot\until\phi$.
\item
\item $\Box(p\imp\Diamond\neg p)\biimp \Box\Diamond\neg p$.
\end{enumerate}
\end{question}

\begin{question}* \
  Further temporal laws: Show that the following are valid.
\begin{enumerate}
\item $\Box(p\imp \next q) \wedge p \imp \Box q$. \ [This law is a
  powerful induction principle if you want to prove that a formula of
  the kind $\Box\phi$ is true.]
\item $(\Box\Diamond p \imp \Box\Diamond q) \ \imp \ \Box(\Box\Diamond
  p \imp \Box\Diamond q)$. \ [This is related to fairness.]
\end{enumerate}
\end{question}

\begin{question} * \
\begin{enumerate}
\item Give a counterexample to $p\until q \ \rightarrow p\before q$.
\item Change the definition of the semantics of $\until$ and $\before$
  in such a way that the formula above becomes valid.
\end{enumerate}
\end{question}

\end{document}


%further questions
\begin{question} 
\begin{enumerate}
\item Give the semantics of the $\until$-operator of LTL as in the
  lecture.
\item Give the semantics of the $U$ operator of SPIN.
\item Comparing $\until$ and $U$, what are the respective advantages
  of the two. 
\end{enumerate}

Define $p\before q$ as $\neg(\neg p \until q)$ and $pBq$ as $\neg(\neg
p U q)$.
\begin{enumerate}
\item Give a counterexample to $p\until q \ \rightarrow
  p\before q$.
\item Argue that $p U q \ \rightarrow \ pBq$ is valid.
\end{enumerate}
\end{question}


\end{document}


