\section{LTL Model Checking with Semantic Tableaux}

To check whether a property $\phi$ holds in a Kripke model $M$, we
`synchronise' the semantic tableaux method for the validity of $\phi$
with the transition system $M$.

\medskip\noindent
Let $M=(X,R,v)$ where $v:X\times\bbP\to\bbtwo$ for a fixed set $\bbP$ of
atomic propositions. Also fix an initial state $x_0\in X$. 
%
We want to build a tableau for $\phi$ and $M$. The nodes $(x,\Phi)$ of
the tableau consist of states $x\in X$ and sets of LTL-formulae
$\Phi$. 

% A node $(x,\Phi)$ is called propositional or transitional if
% $\Phi$ is so called. 

\medskip\noindent The root is $(x_0,\{\neg\phi\}\cup \{p\mid
v(x_0)(p)=1\}\cup\{\neg p \mid v(x_0)(p)=0\} )$.

\medskip\noindent For a non-state node $\Phi$, the children of
$(x,\Phi)$ are $(x,\Psi)$ where $\Psi$ is a child of $\Phi$ according
to the three rules for propositional nodes from above.

\medskip\noindent For state nodes $\Phi$, $(x,\Phi)$ is a leaf (ie, it
has no child) if $\Phi$ contains a contradiction or if $\Phi'$ is
empty or if the successors $(x',\Phi')$ of $(x,\Phi)$ appear on a path
from the root to $(x,\Phi)$. 
%
Otherwise the children of $(x,\Phi)$ are
%
$$(y,\Phi'\cup\{p\in\bbP \mid
v(y,p)=1\}\cup \{\neg p\in\bbP \mid v(y,p)=0\})$$ 
%
for all successors $y$ of $x$, that is, for all $y$ such that $xRy$.


% \medskip\noindent In analogy to the previous section, we say that if
% $\Phi$ has a subsuming ancestor $\Gamma$, then the path from
% $(x,\Gamma)$ to $(x,\Phi)$ is the loop associated with
% $(x,\Phi)$.

% \medskip As before, a leaf $(x,\Phi)$ is called \emph{open} (a) if
% $\Phi$ does not contain a contraction or (b) if there is a loop
% associated with $(x,\Phi)$ that is self-fulfilling. A leaf is called
% \emph{closed} if it is not open.

% \medskip\noindent $M,x_0\models\phi$ if there is a complete tableau
% with root $(x_0,\{\neg\phi\})$ that has no open leaf.  Conversely, if
% there is a complete tableau with root $(x_0,\{\neg\phi\})$ that has an
% open leaf, then the path from the root to that leaf (possibly extended
% by the associated loop) is a counter-example to $\phi$. In Spin, the
% associated loop is called \emph{acceptance cycle}.

%%% Local Variables: 
%%% mode: latex
%%% TeX-master: "sr-logic"
%%% End: 
